\documentclass[a4paper,12pt]{report}

\input{packages} %Подключаем модуль пакетов
\input{styles} %Подключаем модуль стилей

\begin{document}

\input{names} %Подключаем модуль переименования некоторых команд

%%%%%%%%%%%%%%%%%%%%%%%%%%%%%%%%%%%%%%%%%%%%%%%%%%%
% % % % % % % % Титульная страница % % % % % % % % 
%%%%%%%%%%%%%%%%%%%%%%%%%%%%%%%%%%%%%%%%%%%%%%%%%%%
\thispagestyle{empty}

\begin{center}
\MakeUppercase{Московский физико-технический институт (государственный университет)} \par
\MakeUppercase{Факультет инноваций и высоких технологий} \par
\MakeUppercase{Кафедра физико-технической информатики} \par
Направление подготовки: 01.04.02 <<Прикладная математика и информатика>>
\end{center}

\vspace{40mm}

\begin{center}
{\bf \large Автореферат диссертации на соискание академической степени магистра по теме:\par\MakeUppercase{<<Исследование статистического предиктора приступа эпилепсии по данным электроэнцефалограмм>>}
\par}
\end{center}

\vspace{70mm}

\begin{tabular}{lr}
{\large Студент:}     &  {\large Луничкин Егор Валериевич} \\
{\large Научный руководитель:}     &  {\large д.ф.-т.н., с.н.с. Орлов Юрий Николаевич} 
\end{tabular}

\vspace{20mm}


\begin{center}
{Москва -- 2019}
\end{center}

%%%%%%%%%%%%%%%%%%%%%%%%%%%%%%%%%%%%%%%%%%%%%%%%%%%
% % % % % % % % Содержание % % % % % % % % 
%%%%%%%%%%%%%%%%%%%%%%%%%%%%%%%%%%%%%%%%%%%%%%%%%%%
\tableofcontents
\clearpage

%%%%%%%%%%%%%%%%%%%%%%%%%%%%%%%%%%%%%%%%%%%%%%%%%%%
% % % % % % % % Глава 1 % % % % % % % % 
%%%%%%%%%%%%%%%%%%%%%%%%%%%%%%%%%%%%%%%%%%%%%%%%%%%

\chapter{Общая характеристика работы}

\section{Актуальность темы исследования}

\textbf{Эпилепсия} (др.-греч. от «схваченный, пойманный, застигнутый»; лат. epilepsia или caduca) известна людям с давнейших времён. Многие поколения врачей и учёных сталкиваются с проблемой предсказания приступов эпилепсии по состоянию пациента, с проблемой дифферециации обычного состояния больного от предприпадочного состояния.

Однако с тех же самых пор, как человек пытается научиться предсказывать эпилептические припадки, перед ним встаёт вопрос, как и по каким признакам это можно делать, какие из них дают наиболее достоверный результат.

Ввиду невозможности предсказывать будущее со 100\% вероятностью, необходимо построить такую предсказательную модель, которая довольно достоверно будет предугадывать возникновение приступа за некоторое время до его начала.

В настоящее время вопросы предсказания приступа эпилепсии остро стоят перед врачами многих стран, поскольку это заболевание является достаточно распространённым и может постигнуть людей различных возрастов, пола, расы и уровня жизни. Основным инструментом диагностики эпилепсии стала ЭЭГ (электроэнцефалограмма) и <<электроэнцефалография>>~-- чтение и расшифровка результатов ЭЭГ. Так как ЭЭГ представляет собой показания многих датчиков, работающих с частотой несколько тысяч значений в секунду, чтение и анализ ЭЭГ зачастую невозможно невооружённым глазом без помощи компьютера.

Основная тема данного исследования~-- обоснование возможности машинного анализа данных электроэнцефалографии, рассмотрения показания многих датчиков как независимых друг от друга временных рядов и попытка представления этих рядов в качестве стационарных с возможностью предсказания эпилептических приступов.

\section{Степень разработанности}

Основная часть работы посвящена исследованию временных рядов, полученных с датчиков электроэнцефалограмм реальных пациентов, на стационарность в зависимости от различных параметров.

\section{Цели и задачи}

\subsection{Цели работы:}

Основной целью исследования является нахождение алгоритма исследования данных, поступающих с датчиков электроэнцефалограммы, который позволял бы предсказывать наступающие приступы эпилепсии за некоторое определённое время до начала приступа вне зависимости от неконсистентности поступающих данных (например, если один из электродов <<отошёл>> и перестал передавать показания).

\subsection{Задачи работы:}

\begin{enumerate}
    \item Исследовать поступающие на вход данные, представить их в виде независимых друг от друга временных рядов.
    \item Попробовать найти закономерности в данных временных рядах, исследовать стационарность ряда для различных $n$, где $n$~-- количество интервалов, на которые разбивается ряд.
    \item Найти значения функций $F_n(x, t_k) = \Pr{(\xi < x)}$ и $\rho_k = \max_x{|F_n(x, t_k) - F_n(x, t_{k+1})|}$, исследовать, является ли $\{\rho_k\}_1^K$ динамической системой.
    \item На основе полученных данных попытаться построить предсказание эпилептического приступа.
\end{enumerate}

\section{Научная новизна}

Научная новизна работы заключается в следующем: впервые предпринята попытка систематизировать показания различных датчиков ЭЭГ и рассмотреть их с математической точки зрения как стационарные временные ряды и динамические системы.

\section{Теоретическая и практическая значимость работы}

Результаты, полученные в результате данной работы, могут быть положены в основу дальнейших теоретический и практических разработок, касаемых исследования эпилепсии и предсказания приступов. В частности, результаты могут быть использованы для понимания стационарности процессов, происходящих в человеческом мозге.

Наработки, полученные в данной работе, могут быть использованы при разработке промышленных алгоритмов для различных государственных и частных медицинский компаний и учреждений.

\section{Методология и методы исследования}

Основным методом, используемым в работе, является представление результатов ЭЭГ в виде (стационарных) временных рядов и дальнейшее исследование этих временных рядов, как было указано выше. В работе использовались реальные данные реальных пациентов, больных эпилепсией.

Исходные данные с нескольких датчиков ЭЭГ выглядят следующим образом:

\begin{figure}[h!]
    \centering
    \includegraphics[scale=0.45]{1.png}
    \caption{Исходные данные}
\end{figure}

%%%%%%%%%%%%%%%%%%%%%%%%%%%%%%%%%%%%%%%%%%%%%%%%%%%
% % % % % % % % Глава 2 % % % % % % % % 
%%%%%%%%%%%%%%%%%%%%%%%%%%%%%%%%%%%%%%%%%%%%%%%%%%%

\chapter{Основное содержание работы}

\textbf{Во введении} обосновывается актуальность исследования, ставятся цели и задачи, определятся предмет и объект исследования, постулируются основные проблемы исследования и способы их решения, описывается научная новизна и практическая ценность работы.

\textbf{В первой главе} вкратце рассматривается история изучения эпилепсии как болезни, различные подходы к классификации данных и попытки предсказать приступы, которые предпринимались в прошлом.

\textbf{Вторая глава} посвящена анализу входных данных, поступающих с различных датчиков ЭЭГ, предварительной обработке и анализу этих данных, описанию используемых алгоритмов и демонстрации кода.

Входные данные с различных данных, представленные в виде списка:

\begin{figure}[h!]
    \centering
    \includegraphics[scale=0.4]{2.png}
    \caption{Исходные данные в виде списков}
\end{figure}

\newpage
График показаний одного датчика (каждое 1000-е измерение):

\begin{figure}[h!]
    \centering
    \includegraphics[scale=0.45]{3.png}
    \caption{График показаний одного датчика}
\end{figure}

\textbf{В третьей главе} проводится основная математическая и алгоритмическая работа над полученными данными. Производится построение динамической системы и строится предиктор эпилептического приступа.

\textbf{Четвёртая глава} посвящена теоретическим выкладкам, в которых производится попытка теоретически обосновать результаты работы каждого шага полученного алгоритма.

И, наконец, \textbf{в заключении} собираются воедино все полученные в данной работе результаты. Описывается алгоритм построения предиктора эпилептического приступа по данным электроэнцефалограмм. Формулируются основные теоретические выводы, полученные в работе, описывается возможное дальнейшее направление работы и способы применения полученных алгоритмов.

%%%%%%%%%%%%%%%%%%%%%%%%%%%%%%%%%%%%%%%%%%%%%%%%%%%
% % % % % % % % Глава 3 % % % % % % % % 
%%%%%%%%%%%%%%%%%%%%%%%%%%%%%%%%%%%%%%%%%%%%%%%%%%%

\chapter{Заключение}

В результате работы были получены следующие важные результаты:

\begin{enumerate}
    \item Были исследованы реальные данные, поступающие с датчиков электроэнцефалограмм реальных пациентов.
    \item Были исследованы временные ряды и соответствующие динамические системы, построена математическая модель.
    \item Был построен предиктор эпилептического приступа.
\end{enumerate}

%%%%%%%%%%%%%%%%%%%%%%%%%%%%%%%%%%%%%%%%%%%%%%%%%%%
% % % % % % % % Список литературы % % % % % % % % 
%%%%%%%%%%%%%%%%%%%%%%%%%%%%%%%%%%%%%%%%%%%%%%%%%%%

\begin{thebibliography}{9}
\addcontentsline{toc}{chapter}{\bibname}

\bibitem{Orlov-bib1}
Ивченко А.Ю., Козлова А.Б., Корсакова М.Б., Машеров Е.Л., Орлов Ю.Н., Руссков А.А. \textit{Анализ нестационарности ЭКоГ и построение предвестника разладки} \\
\textbf{Препринты ИПМ им. М.В. Келдыша.} 2017. № 49. 19 с.

\end{thebibliography}
\end{document}
